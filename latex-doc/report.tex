\documentclass[a4paper, 11pt]{article}

\usepackage[utf8]{inputenc}
\usepackage[T1]{fontenc}
\usepackage[brazilian]{babel}
\usepackage{graphicx}
\usepackage{amsmath}
\usepackage{geometry}
\usepackage{tikz}
\usetikzlibrary{arrows.meta, positioning}
\usepackage{subcaption}
\geometry{a4paper, margin=1in}

\title{Relatório Técnico: Sistema de Controle Ball-Beam com PID}
\author{
\begin{tabular}{ll}
\textbf{Aluno:} & César Henrique Almeida Kerber \\
\textbf{RA:} & SJ3018873 \\
\\
\textbf{Aluno:} & Lucas Camargo Ekroth \\
\textbf{RA:} & SJ3017532 \\
\\
\textbf{Aluno:} & Paulo Ricardo Rotband Rangel \\
\textbf{RA:} & SJ3018334 \\
\end{tabular}
}
\date{\today}

\begin{document}

\maketitle

\begin{abstract}
Este relatório detalha o projeto, implementação e análise de um sistema de controle Proporcional-Integral-Derivativo (PID) para a estabilização de uma esfera em uma viga, conhecido como sistema "Ball-Beam". O objetivo principal é manter a esfera em uma posição de referência ajustável, utilizando um servo motor para inclinar a viga e um sensor ultrassônico para medir a posição da esfera. A interface de controle foi desenvolvida em Python com Streamlit, e a comunicação com o hardware (Arduino) foi realizada via Telemetrix. A modelagem matemática do sistema é apresentada, seguida pela descrição da implementação do controlador e a análise dos resultados obtidos, que foram impactados pelo ruído no sensor. Por fim, são discutidas as conclusões e possíveis melhorias futuras para o projeto.
\end{abstract}

\tableofcontents
\newpage

\section{Introdução}
Este projeto tem como objetivo implementar um controlador PID para um sistema de controle de posição de uma bola em uma viga (Ball-Beam). O sistema "Ball and Beam" é um problema clássico e amplamente utilizado em laboratórios de automação e controle para a demonstração de diversas técnicas, desde o controle PID até abordagens mais avançadas como controle robusto, adaptativo e baseado em inteligência artificial.

O objetivo específico deste trabalho é desenvolver um sistema de controle PID para estabilizar uma bola em uma viga, utilizando um motor de servo para ajustar a inclinação da viga. O sistema deve ser capaz de manter a bola em uma posição desejada, mesmo com perturbações externas.

\section{Descrição do Sistema}
O sistema consiste em uma viga onde uma esfera pode rolar livremente. A inclinação da viga é controlada por um servo motor, que por sua vez altera a posição da esfera. A posição da esfera é medida por um sensor de distância ultrassônico.

\subsection{Materiais Utilizados}
\begin{itemize}
    \item 1 Arduino Uno
    \item 1 Servo Motor
    \item 1 Sensor de Distância Ultrassônico
    \item 1 Esfera pequena e leve
    \item 1 Estrutura de viga (impressa em 3D)
\end{itemize}

\subsection{Diagrama de Conexões}
A figura a seguir ilustra as conexões elétricas entre o Arduino, o servo motor e o sensor ultrassônico.
\begin{figure}[h!]
    \centering
    % Placeholder para a imagem, já que não posso acessar o arquivo local
    \framebox[0.8\textwidth]{Placeholder para a imagem: 'Pasted image 20250617190928.png'}
    \caption{Diagrama de conexões do hardware.}
    \label{fig:conexoes}
\end{figure}

\section{Modelagem do Sistema}
A modelagem do sistema Ball-Beam é um passo crucial para o desenvolvimento do controlador. O sistema pode ser modelado como um sistema de segunda ordem, com a posição da bola ($x$) como a variável de saída e o ângulo da viga ($\theta$) como a variável de entrada.

A equação diferencial que descreve o movimento da bola na viga, ignorando o atrito, é dada por:
$$
\ddot{x} = \frac{m \cdot g \cdot \sin(\theta)}{m + \frac{I}{r^2}}
$$
Onde:
\begin{itemize}
    \item $x$ é a posição da bola
    \item $m$ é a massa da bola
    \item $g$ é a aceleração da gravidade
    \item $\theta$ é o ângulo da viga
    \item $I$ é o momento de inércia da bola
    \item $r$ é o raio da bola
\end{itemize}

Para pequenos ângulos, $\sin(\theta) \approx \theta$. Aplicando a transformada de Laplace, a função de transferência do sistema pode ser obtida:
$$
G(s) = \frac{X(s)}{\Theta(s)} = \frac{m \cdot g \cdot L}{s^2(m \cdot L^2 + I)}
$$
Onde $L$ é o comprimento da viga.

Apesar da modelagem teórica, devido à instabilidade natural do sistema e ao alto ruído presente no sinal de feedback do sensor ultrassônico, foi impossibilitada a aquisição de dados experimentais para a identificação precisa do sistema e validação do modelo.

\section{Implementação do Sistema de Controle}
\subsection{Software e Interface}Utilizamos Python para a implementação do controlador PID e para a criação de uma Interface de Controle e Configuração do Sistema. A biblioteca Streamlit foi usada para desenvolver a interface gráfica (IHM), permitindo ao usuário definir o setpoint (posição desejada) e ajustar os ganhos do PID em tempo real. A comunicação com o hardware (Arduino) foi gerenciada pela biblioteca Telemetrix, que utiliza uma versão do protocolo Firmata. A comunicação entre o backend (controlador) e o frontend (interface) foi realizada via API REST.\subsection{Interface Gráfica (IHM)}A interface gráfica desenvolvida com Streamlit permite a fácil interação com o sistema. O usuário pode ajustar os parâmetros do controlador PID e o setpoint, e observar o comportamento do sistema através de gráficos em tempo real.\begin{figure}[h!]    \centering    \begin{subfigure}[b]{0.45\textwidth}        \centering        \framebox[\textwidth]{Placeholder para a imagem: 'Interface do Streamlit - Controles PID'}        \caption{Controles para ajuste dos ganhos PID e do Setpoint.}        \label{fig:streamlit_controles}    \end{subfigure}    \hfill    \begin{subfigure}[b]{0.45\textwidth}        \centering        \framebox[\textwidth]{Placeholder para a imagem: 'Interface do Streamlit - Gráfico'}        \caption{Gráfico de posição da bola vs. tempo.}        \label{fig:streamlit_grafico}    \end{subfigure}    \caption{Interface de controle e monitoramento desenvolvida com Streamlit.}    \label{fig:streamlit_interface}\end{figure}

\subsection{Diagrama de Blocos}
O diagrama de blocos a seguir representa a malha de controle do sistema. O setpoint é comparado com a posição atual da bola (medida pelo sensor), e o erro resultante alimenta o controlador PID. A saída do PID atua no servo motor, que ajusta o ângulo da viga, fechando a malha.

\begin{tikzpicture}[auto, node distance=2cm, >=Stealth]

    % Nós do diagrama
    \node[coordinate] (input) {};
    \node[circle, draw, minimum size=0.8cm, right=of input] (sum) {\scriptsize$+$\hspace{-0.8ex}\raisebox{-1ex}{$-$}};
    \node[block, draw, rectangle, right=of sum] (pid) {PID};
    \node[block, draw, rectangle, right=of pid] (servo) {Servo};
    \node[coordinate, right=of servo] (output) {};
    \node[block, draw, rectangle, below=1.5cm of pid] (sensor) {Sensor Ultrassônico};

    % Ligações do diagrama
    \draw[->] (input) -- node[near start] {Entrada (cm)} (sum);
    \draw[->] (sum) -- (pid);
    \draw[->] (pid) -- (servo);
    \draw[->] (servo) -- node[near end] {Saída} (output);

    \draw[->] (servo.east) -- ++(0.5,0) |- (sensor);
    \draw[->] (sensor) -| (sum);

\end{tikzpicture}

\subsection{Implementação do Controlador PID}
A implementação do controlador PID foi realizada em Python. O controlador calcula a ação de controle com base nos ganhos Proporcional ($K_p$), Integral ($K_i$) e Derivativo ($K_d$). Devido ao ruído significativo no sinal de feedback do sensor, os parâmetros do PID foram ajustados de forma empírica, buscando um compromisso entre a resposta do sistema e a estabilidade.

\section{Resultados}
Após a implementação, o sistema foi testado em diferentes condições de operação. Os resultados mostraram que o sistema não teve a capacidade de controlar a posição da bola com alta precisão e estabilidade. A principal causa foi o ruído excessivo e a baixa resolução do sensor ultrassônico, que introduzia variações significativas na leitura da posição, levando a uma atuação instável do controlador.

Apesar disso, foi possível observar que o sistema foi capaz de reagir às mudanças de posição da bola e tentou mantê-la em torno do ponto de equilíbrio (setpoint), demonstrando o funcionamento da lógica de controle.

\section{Conclusão e Melhorias Futuras}
Este projeto demonstrou a viabilidade de controlar a posição de uma bola em uma viga usando um controlador PID e hardware de baixo custo. No entanto, a qualidade do controle foi severamente limitada pela precisão do sensor de distância. O ruído no sinal de feedback impediu um ajuste fino dos ganhos do PID e a obtenção de um desempenho estável e preciso.

Como melhorias futuras, sugere-se:
\begin{itemize}
    \item \textbf{Substituição do Sensor:} Utilizar um sensor de distância de maior precisão e menor ruído, como um sensor a laser ou um sistema baseado em visão computacional com uma câmera.
    \item \textbf{Filtragem de Sinal:} Implementar filtros digitais (e.g., filtro de média móvel ou filtro de Kalman) no sinal do sensor para reduzir o ruído antes de alimentar o controlador.
    \item \textbf{Controle Avançado:} Com um sinal de feedback mais limpo, seria possível implementar estratégias de controle mais avançadas para melhorar a estabilidade e a precisão do sistema.
\end{itemize}

\section{Referências}
\begin{thebibliography}{9}
    \bibitem{nise} Nise, N. S. (2010). \textit{Control Systems Engineering}. Wiley.
    \bibitem{saad} Saad, M., & Khalallah, M. (2017). Design and implementation of an embedded ball-beam controller using PID algorithm. \textit{Universal Journal of Control and Automation}, 5(4), 63-70.
    \bibitem{streamlit} Streamlit Documentation. \texttt{https://docs.streamlit.io/}
    \bibitem{telemetrix} Telemetrix Documentation. \texttt{https://mryslab.github.io/telemetrix/}
    \bibitem{wolfram} Wolfram Ball And Beam Control Document. \texttt{https://reference.wolfram.com/language/MicrocontrollerKit/workflow/BallAndBeamControl}
\end{thebibliography}

\end{document}
